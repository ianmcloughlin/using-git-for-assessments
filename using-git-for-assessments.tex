\documentclass[12pt, a4paper, hidelinks]{scrartcl}

  % Enables the use of colour.
  \usepackage{xcolor}
  % Syntax high-lighting for code. Requires Python's pygments.
  \usepackage{minted}
  % Enables the use of umlauts and other accents.
  \usepackage[utf8]{inputenc}
  % Diagrams.
  \usepackage{tikz}
  % Settings for captions, such as sideways captions.
  \usepackage{caption}
  % Symbols for units, like degrees and ohms.
  \usepackage{gensymb}
  % Latin modern fonts - better looking than the defaults.
  \usepackage{lmodern}
  % Allows for columns spanning multiple rows in tables.
  \usepackage{multirow}
  % Better looking tables, including nicer borders.
  \usepackage{booktabs}
  % More math symbols.
  \usepackage{amssymb}
  % More math fonts, like \mathbb.
  \usepackage{amsfonts}
  % More math layouts, equation arrays, etc.
  \usepackage{amsmath}
  % More theorem environments.
  \usepackage{amsthm}
  % More column formats for tables.
  \usepackage{array}
  % Adjust the sizes of box environments.
  \usepackage{adjustbox}
  % Better looking single quotes in verbatim and minted environments.
  \usepackage{upquote}
  % Better blank space decisions.
  \usepackage{xspace}
  % Better looking tikz trees.
  \usepackage{forest}
  % URLs.
  \usepackage{hyperref}
  % Plotting.
  \usepackage{pgfplots}
  % Changing headers and footers.
  \usepackage{fancyhdr}
  % Calculates the number of pages.
  \usepackage{lastpage}
  % Styling the abstract.
  \usepackage{abstract}
  
  % Various tikz libraries.
  % For drawing mind maps.
  \usetikzlibrary{mindmap}
  % For adding shadows.
  \usetikzlibrary{shadows}
  % Extra arrows tips.
  \usetikzlibrary{arrows.meta}
  % Old arrows.
  \usetikzlibrary{arrows}
  % Automata.
  \usetikzlibrary{automata}
  % For more positioning options.
  \usetikzlibrary{positioning}
  % Creating chains of nodes on a line.
  \usetikzlibrary{chains}
  % Fitting node to contain set of coordinates.
  \usetikzlibrary{fit}
  % Extra shapes for drawing.
  \usetikzlibrary{shapes}
  % For markings on paths.
  \usetikzlibrary{decorations.markings}
  % For advanced calculations.
  \usetikzlibrary{calc}
  
  % GMIT colours.
  \definecolor{gmitblue}{RGB}{20,134,225}
  \definecolor{gmitred}{RGB}{220,20,60}
  \definecolor{gmitgrey}{RGB}{67,67,67}

  %%%%%% CHANGE
  % Author.
  \newcommand{\docauthor}{ian.mcloughlin@gmit.ie}
  % Title.
  \newcommand{\doctitle}{Using git for assessments}
  %%%%%%

  
  % Abstract
  \renewcommand{\abstractname}{}
  \renewcommand{\absnamepos}{empty}

  % Header and footer
  \fancypagestyle{plain}{
    \renewcommand{\headrulewidth}{0pt}
    \lhead{}
    \rhead{}
    \cfoot{Page~\thepage~of~\pageref{LastPage}}
    \renewcommand{\footrulewidth}{0.1pt}
  }
  \pagestyle{fancy}
  \fancyhf{}
  \lhead{\docauthor}
  \rhead{\doctitle}
  \cfoot{Page~\thepage~of~\pageref{LastPage}}
  \renewcommand{\headrulewidth}{0.1pt}
  \renewcommand{\footrulewidth}{0.1pt}

  % Tables
  \newcolumntype{x}[1]{>{\centering\arraybackslash\hspace{0pt}}p{#1}}

  % Bibliography
  \renewcommand{\refname}{\selectfont\normalsize References} 

  \title{\doctitle}
  \author{\docauthor}
  \date{}

\begin{document}
  
\maketitle

\noindent
These are the instructions for completing the assessments for this module using git~\cite{git}.
\subsection*{Why we are using git?}
  \begin{itemize}
    \item It is the industry standard for tracking the development of code, text, and software.
    \item It enables students to demonstrate consistency in their work.
    \item It provides evidence that a student's work is their own.
  \end{itemize}

  \subsection*{How do I submit my assessments?}
  \begin{itemize}
    \item Each assessment should be in a single git repository and regularly synced with a hosting platform like GitHub~\cite{github}.
    \item The URL for the online repository should be submitted using the form on the Moodle page.
    \item The URL must be accessible by the lecturer. If it is not publicly available, the lecturer must be added as a collaborator.
    \item Your last commit synced to the online hosting platform before the assessment deadline will form your submission.
    \item Once you are given an assessment, you should immediately create the repository and submit its URL. There is no benefit in delaying.
  \end{itemize}

  \subsection*{Can you give me a quick overview?}
  \begin{itemize}
    \item Imagine I give you a PDF describing the requirements of a project to be completed within the next six weeks.
    \item You should go to GitHub, create a new repository for the project with a descriptive name under your own account. When you create the repository, copy its URL from the location bar of your web browser.
    \item Go to the Moodle page and click the link for the project submission form. Paste your URL into the form and click Submit.
    \item Clone the repository to your computer and think about a realistic plan to break your project into reasonable commits, maybe three or four each week.
  \end{itemize}

  \subsection*{What is a reasonable commit?}
  \begin{itemize}
    \item There is no perfect commit, but there are plenty of bad ones. It helps to imagine you are working on the repository as part of a team.
    \item A good commit does not change many different files. This could cause conflicts for you and other team members working on the same files.
    \item Likewise, making lot of small commits in quick succession will annoy your teammates. For instance, you should bunch README typo fixes into one commit.
    \item A reasonable commit might be a well-commented function to calculate the factorial of a number or an important bug fix like changing an erroneous ``\texttt{=}'' to ``\texttt{==}''.
  \end{itemize}

  \subsection*{Any advice from past submissions?}
    \begin{itemize}
      \item Go easy on yourself but do the work.
      \item Everyone is susceptible to procrastination and disorganisation.  You are expected to be aware of this and take reasonable measures to avoid them.
      \item Using work from outside sources is acceptable so long as you clearly reference the source and the overall submission is substantially your own work.
      \item You must be able to explain your project during and after its completion. Our academic policies relating to plagiarism and conduct are available on the website~\cite{gmitqaf}. 
      \item Unpreventable problems sometimes arise. It really helps if you can demonstrate that, up until that point, you had completed a proportionate amount of work.

  \end{itemize}


  \bibliographystyle{plain}
  \bibliography{bibliography}
\end{document}
